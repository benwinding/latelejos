%%%%Document for Project readiness client meeting
\documentclass[11pt, letterpaper]{article}
\usepackage{times}
\usepackage{ifthen}
\usepackage{amsmath}
\usepackage{amssymb}
\usepackage{graphicx}
\usepackage{setspace}

\setlength{\parindent}{0pt}
\addtolength{\oddsidemargin}{-2 cm}
\addtolength{\textwidth}{3 cm}

\title{Project Readiness Overview}
\begin{document}
\maketitle

\section{Project Roles}

\begin{itemize}
\item Project Manager: Pavi
\item Document Manager: Ben 
\item Coding Manager: tbd
\item Coding and Documenation: All inclusive
\item Testing: All members to write at least one test script
\end{itemize}


\section{Group Strengths}
\subsection{Coding Background}
\begin{itemize}
\item Several members have previous in-depth coding experience.
\item All members have worked on group projects previously.
\item At least one member has worked in industry for over a year.
\item Group members' are attitudes are cohesive and personable.
\end{itemize}

\section{Process Model}
\subsection{Model choice}
The team will use a mixture of agile method and waterfall method.\par
The Waterfall method structure will be used to provide an idea for internal milestones to set for the team and 
to allow the meeting of documentation and software goals.\par
The Agile method will be used to allow for increased productivity from an experienced team 
and allow continuous and frequent delivery of software to the client.
\subsection{Model choice rationale}
Waterfall model: 
\begin{itemize}
\item Facilitates production of complete documentation during the life of the project.
\item Requirements are unlikely to change significantly throughout the project
\item Structure is easy to adhere to and allows a template for internal milestones
\end{itemize}
Agile method: 
\begin{itemize}
\item Allows experimentation and quick production of working software, 
this is advantageous as we are working with software libraries with which we
 have no previous experience.
\item Easy to track work done and create short term milestones
\item Takes advantage of team's coding experience
\item Allows quick response to any necessary requirements changes
\end{itemize}

\section{Git Organization}

\section{Distribution of Labor}
\subsection{Meetings}
The internal and client meetings will have a rotating roster for Chair person, Minute taker, 
Time keeper and Facilitator.
\subsection{Coding}
Coding the first part of the project has been split into two parts: Robot and GUI.
The Robot portion will include motor and sensor method implementation. 
\par This will be handled by:
\begin{itemize}
\item Sammy
\item John
\item Pavi
\item Issac
\end{itemize}
While the GUI portion will handle the representation of the remote control and the map.\par
This will be handled by:
\begin{itemize}
\item Ben
\item Sean
\end{itemize}
\subsection{Documentation}
Documentation creation and proofreading shall be shared on a rotating roster.
Documentation content can be generated via brainstorm sessions.
All deliverable documentation is to be written in LaTex and outputted to pdf for client.

\section{Leadership}
Responsibilites for milestones will be allocated individually based on experience and competency.
Final decisions for project deliverables will be made via discussion and general consensus or, in 
the case of disagreement, majority vote.
Project Manager will be responsible for checking on general team progress.\par
Documentation Manager will be responsible for proof reading final documentation to client and possibly
collating seperate documentation components into single document.\par
Coding manager will be responsible for ensuring coding is kept to consistent standard and functional.\par
Testing manager may be required for ensuring testing is being written and completed successfully.

\end{document}
