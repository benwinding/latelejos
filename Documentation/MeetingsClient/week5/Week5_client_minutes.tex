%%%%%%weekly meeting template, prepared by Michael Sheng.  09/03/2007
\documentclass[11pt, a4paper]{article}
\usepackage{times}
\usepackage{ifthen}
\usepackage{amsmath}
\usepackage{amssymb}
\usepackage{graphicx}
\usepackage{setspace}

%%% page parameters
\oddsidemargin -0.5 cm
\evensidemargin -0.5 cm
\textwidth 15 cm
\topmargin -1.2 cm
\textheight 22 cm

\renewcommand{\baselinestretch}{1.4}\normalsize
\setlength{\parskip}{0pt}

\begin{document}
%%%mention the no, time, and venue of the meeting
\noindent Software Engineering Group Project PG 29 {\bf Ingkarni Wardli 462} at {\bf 3:40 pm Tuesday 22/08/2017}.
\vspace*{10pt}
\begin{center}
\huge Week 5, 3rd Client Meeting Minutes
\end{center}
\vspace*{10pt}
\begin{center}
\huge Prepared by Sean
\end{center}
\begin{center}
\huge August 22, 2017
\end{center}
%%%first, nominate a chair for the meeting. We suggest that each member at least has one chance as the chair.
\section*{Chair: Ben}

\section{Attendees}
\begin{itemize}
\item Chair: Pavi
\item Facilitator: Ben
\item TimeKeeper: Sammy
\item Recorder: Sean
\item Issac
\item Huy Nguyen
\end{itemize}

%%%if some students cannot make the meeting due to some reasons, their names should appear here.


\section{Apologies}
%\begin{itemize}

%\end{itemize}


\section{Q\&A with tutor}

\begin{itemize}
\item what happens when switching back to auto after Manual override? A: up to group, continue from last known point or resume survey from new point
\item Must be able to manual override at anytime and switch back
\item Manual mode must have warning when obstacle or NGZ detected
\item survey timelimit: 15-20 minutes
\item robot is allowed in radiation area

\end{itemize}

\section{Q \& A section with Client}
\begin{itemize}
\item Milestones: on track to finish robot by week 9
\item next week: expect robot tracking on map during manual control
\end{itemize}

\section{SRS Feedback}
\begin{itemize}
\item version number
\item grammar and spelling
\item remove unnecessary figures
\item sec 2.3 check priority
\item missing \% signs
\item more detailed technical information required in some sections
\item more assumptions e.g: battery life, work force changes, robot will not be surrounded by NGZs
\item Requirement ID numbers
\item priority explanation: high medium low, add to document convention
\item requirement status: complete, pending, waiting for other, etc..
\item distinguish between funcitonal and non functional requirements in some sections. Eg Sensors: crater detection vs. must not enter crater
\item Map Implementation UI section needs more structure
\item Review GUI: larger map, control and map swap sides, add help button
\end{itemize}
%%%any schedules for this meeting should go next, each with a separate section.

%%%more issues should make it like the above one.

%%%finally, specifies time of next meeting
%\section{Next meeting}
\begin{itemize}
\item Date: 12pm-2pm, Friday, 18 Aug
\item Location: tbd
\item Purpose: review SRS
\end{itemize}

\vspace*{10pt}

\end{document}