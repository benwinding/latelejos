\documentclass[10pt,a4paper,titlepage]{article}
\usepackage[utf8]{inputenc}

\usepackage{amsmath}
\usepackage{amsfonts}
\usepackage{amssymb}
\usepackage{graphicx}
\usepackage{float} % force figure to render inline location
\usepackage{enumitem} % apt install texlive-latex-extra 
\usepackage{anyfontsize} % custom fontsizes
\usepackage{titlesec} % custom section spacings
\usepackage{multirow} % merge table rows
\usepackage{vhistory} % revision table package
\usepackage{pdfpages}

\setlist[itemize]{noitemsep} % No spaces in itemize lists
\setlist[enumerate]{noitemsep} % No spaces in itemize lists
\setlist[description]{noitemsep} % No spaces in itemize lists
\titlespacing*{\subsubsection}{0pt}{8pt}{2pt}
\titlespacing*{\paragraph}{0pt}{3pt}{5pt}

\newcommand{\cpright}{\textsuperscript{\tiny\copyright}}

\setlength\parindent{0pt}

\begin{document}
	
	\begin{titlepage}
		
		\title{
			\fontsize{50}{12}\selectfont{\textsc{Lunar Rover}}\\
			\vspace{20pt}
			\fontsize{20}{12}\selectfont{\textsc{User Manual}}\\
			\vspace{10pt}
			\large{Software Engineering \& Project} \\
			\vspace{20pt}
			\includegraphics[width=200px]{title-page-ev3.png}					
		}
		\date{17/10/2017}
		\author{
			\bf{Team: PG-29} \\
			Benjamin Winding \\
			Kin Leong Lee \\
			Pavitterjeet Sidhu \\
			Phan Huy Nguyen \\
			Sean Hennessy \\
			Xiaoshan Chen \\
		}
		\maketitle
		
	\end{titlepage}
		 
	\tableofcontents	
	
	
	
	\section*{Revision History}	
	\label{revtable}	
	\begin{tabular}{|p{2.1cm}|p{2.5cm}|p{2cm}|p{4.1cm}|}		
		\hline 
		\textbf {Name} & \textbf{Date} & \textbf {Version} &\textbf {Summary of Changes} \\ 
		\hline 
		\hline 		
	\end{tabular}

	\newpage
	
	\section{GENERAL INFORMATION}
		\subsection{System Overview}
        This system consist two components a) The LEGO Mindstorm robot and the software to control it. The main aim of this system is to survey the specified area and safely return back to landing zone. The system is fully developed, tested and operational. GUI is also included in the system in-order to control the robot.
        The main features of this robot include
	\begin{itemize}
		\item Automatic survey of specified areas
		\item Remote control manual override and movement
		\item On-board obstacle avoidance mechanisms 
		\item No-go zone detection and avoidance
		\item Ability to return to the starting point or any point selected on mapped area.
\end{itemize}
   

        \subsection{Organization of the Manual}
        \subsubsection{User’s Manual v1.0.}
        \begin{itemize}
		\item Section 1 includes general information about the system
		\item Section 2 includes system summary
		\item Section 3 includes how to set up the system
		\item Section 4 includes how to use the system
\end{itemize}
        \subsection{Definitions, Acronyms and Abbreviations}
        \subsubsection{Definitions}
\begin{itemize}
\item Intellij IDE - Java(IDE) for developing computer software
\end{itemize}

\subsubsection{Acronyms}
\begin{itemize}
\item RMS - Robot Mapping System
\item OS - Operating System
\item JRE - Java Runtime Environment
\item IDE - Integrated development environment
\item NGZ - No Go Zone
\item SDD - Software Design Document
\item SRS - Software Requirements Specification
\item WDDM - Windows Display Driver Model
\end{itemize}

\subsubsection{Abbreviations}
\begin{itemize}
	\item min - minute
\end{itemize}
    \newpage    
	\section{SYSTEM SUMMARY}
		\subsection{System Configuration}
        The system features a GUI which can be used to instruct the robot. The user can simple click the connect button to connect the robot through the wifi but the only condition is that the  computer and the robot should be connected to the same network. User can load a new survey map to the system using the load map button on GUI. User can also set NGZ using corresponding button. 
        \subsection{Data Flows}
        User can provide input using the keyboard of the computer and using the mouse as well. The information such as robot's location on the survey map, data from the sensors of the robot will be visible on the GUI in real time.
        \subsection{User Access Levels}
        The is no security and authentication level included in the system.
        \subsection{Contingencies }
        The only factor which may degrade the performance of the system is the network speed. In the case of an emergency during the survey, a stop button is included in the GUI which will immediately stop the robot. 
    \newpage
	\section{GETTING STARTED}
	\newpage
	\section{USING the SYSTEM}
	\newpage
	\section{Appendices}
		
	
\end{document}
