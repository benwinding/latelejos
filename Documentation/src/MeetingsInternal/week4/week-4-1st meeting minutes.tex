%%%%%%weekly meeting template, prepared by Michael Sheng.  09/03/2007
\documentclass[11pt, a4paper]{article}
\usepackage{times}
\usepackage{ifthen}
\usepackage{amsmath}
\usepackage{amssymb}
\usepackage{graphicx}
\usepackage{setspace}

%%% page parameters
\oddsidemargin -0.5 cm
\evensidemargin -0.5 cm
\textwidth 15 cm
\topmargin -1.2 cm
\textheight 22 cm

\renewcommand{\baselinestretch}{1.4}\normalsize
\setlength{\parskip}{0pt}

\begin{document}
	%%%mention the no, time, and venue of the meeting
	\noindent Software Engineering Group Project PG 29 {\bf Hub Room 339} at {\bf 12 pm Friday 18/08/2017}.
	\vspace*{10pt}
	\begin{center}
		\huge Week 4, 1st Meeting Minutes
	\end{center}
	\vspace*{10pt}
	\begin{center}
		\huge Prepared by Huy Nguyen
	\end{center}
	\begin{center}
		\huge August 18, 2017
	\end{center}
	%%%first, nominate a chair for the meeting. We suggest that each member at least has one chance as the chair.
	\section*{Chair:}
	
	\section{Attendees}
	\begin{itemize}
		\item Sean
		\item Sammy
		\item Pavi
		\item Isaac
		\item Ben
		\item Huy Nguyen
	\end{itemize}
	
	%%%if some students cannot make the meeting due to some reasons, their names should appear here.
	
	
	\section{Meeting Outcome}
	 Separate SRS document task for members
	\begin{enumerate}
		\item Introduction (BEN)
		\begin{itemize}
			\item Purpose of document 
			\begin{itemize}
				\item Detail overview of the product
				\item Target audience
				\item User Interface
				\item Hardware and software requirement
				\item As Describe by project specs 
			\end{itemize}
			\item Document Covention
			\begin{itemize}
				\item Special notation (ex: App for Application)
				\item Special symbols
				\item Highlighting convention
			\end{itemize}
			\item Target audience
			\begin{itemize}
				\item Client description
				\item Project team
				\item Management team
				\item Testing team
				\item ....
			\end{itemize}
			\item  Project Scope
			\begin{itemize}
				\item Reflect the specs
				\item Any unique expectation
			\end{itemize}
			\item References
			\begin{itemize}
				\item Any refernce document use in this project
			\end{itemize}
		\end{itemize}
		\item Overall Description (Sami)
		\begin{itemize}
			\item Project perspective
			\begin{itemize}
				\item New self contained project
				\item Relies on existing lego robot
				\item Diagram, connection and interfaces
			\end{itemize}
		
			\item Product features
			\begin{itemize}
				\item Summary features
				\item Survey area and Mapping the survey area is the most important
				\item Manual controliv.	Autonomous control
				\item Obstacle avoidance
				\item NGZ detection
				\item Start from landing point then Survey then Return
			\end{itemize}
		
			\item User classes and characteric
			\begin{itemize}
				\item Who will use this application (Student, teacher)
				\item Only basic knowledge of GUI control
			\end{itemize}
		
			\item Operating Environment 
			\begin{itemize}
				\item Window 7 or above
				\item Mac OS 10 or above
				\item Linus
				\item Ubuntu 16+
				\item Hardware:
				\begin{itemize}
					\item Windows: 1 gigahertz (GHz) or faster 32-bit (x86) or 64-bit (x64) processor*
					1 gigabyte (GB) RAM (32-bit) or 2 GB RAM (64-bit)
					16 GB available hard disk space (32-bit) or 20 GB (64-bit)
					DirectX 9 graphics device with WDDM 1.0 or higher driver.
					\item Mac:
					An Intel Core 2 Duo, Core i3, Core i5, Core i7, or Xeon processor.
					Mac OS X v10.6.6 or later to install via the Mac App Store (v10.6.8 recommended)
					7 GB of available disk space.
					2 GB of RAM.
				\end{itemize}
			\end{itemize}
		
			\item Design and Implementation Constrains
			\begin{itemize}
				\item Tools
				\item Minimum hardware/soft requirements
				\item LeJos 0.9.0
				\item Java version 7
				\item Wifi (encryption)
				\item Leo Kit
				\item 10 percent external code max
			\end{itemize}
		
			\item User Documentation
			\begin{itemize}
				\item Describe about User manual
			\end{itemize}
		
			\item Assumptions
			\begin{itemize}
				\item	Any landing zones
				\item Clear colors of track/trails and the map that
				\item Flat surface
				\item Thickness of the line making (2cm)
				\item Trail reasonable smooth
				\item Survey area is rectangular area.
				\item Safe landing zone ( it can not NGZ or and danger area)
			\end{itemize}
		\end{itemize}
	
		\item User Requirements (Pavi)
		\begin{itemize}
			\item Survey the are and show map
			\item GUI for manual control and map
			\item Able to draw NGZ  on GUI
			\item Set start and destination point
			\item Obstacle avoid
			\item Switch between auto and manual
			\item NGZ avoidance
			\item Stop (emergency stop)
			\item Robot must return to pickup zone
			\item Track detection
			\item Recognize different obstacles
			\item Take partial map and complete the map
			\item Map can be recorded to XML format
			\item Map has to be zoomable 
			\item No more than ½ of the robot can cross the NGZ/boundary/obstacle/hazardous area
			\item Map construct in Realtime include obstacle just discover.
			\item Map track position of robot
			\item Must finish survey in reasonable time.
			\item How to security the robot from taking control from un-authorize person
		\end{itemize}
	
		\item  System Feature (Huy Nguyen)
		\begin{itemize}
			\item GUI: Map and relevant controller
			\begin{itemize}
				\item Manual controller
				\begin{itemize}
					\item Direction buttons and keyboard control
					\item Stop button
					\item Connect button
					\item User authentication (Username/password)
					\item Switch controller mode button.
				\end{itemize}
				\item Auto controller
				\begin{itemize}
					\item Stop button
					\item Auto controller
				\end{itemize}
				\item Map
				\begin{itemize}
					\item Display survey area on map
					\item Display robot moment on map
					\item Use mouse to define NGZ
					\item Load and save map
				\end{itemize}
			\end{itemize}
			\item Robot:
			\begin{itemize}
				\item Color sensor recognization
				\item Wifi: connection
				\item Ultra sonic sensor recognition
			\end{itemize}
		\end{itemize}
	
		\item External Interface Requirement (Issac)
		\item None Function Requirement
		\begin{itemize}
			\item Performance
			\begin{itemize}
				\item Accuracy
				\item Fast survey
				\item Easy to use
				\item Stable and high speed internet connection
			\end{itemize}
			\item Safety Requirement 
			\begin{itemize}
				\item No high impact collation
				\item No more than ½ NGZ
				\item Return landing zone 
			\end{itemize}
			\item Security requirement
			\begin{itemize}
				\item Avoid un-authorize access
				\item ...
			\end{itemize}
		\end{itemize}
	\end{enumerate}
	
	%%%any schedules for this meeting should go next, each with a separate section.
	
	%%%more issues should make it like the above one.
	
	%%%finally, specifies time of next meeting
	\section{Next meeting}
	
	\vspace*{10pt}
	
\end{document}